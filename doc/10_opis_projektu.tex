\section {Opis projektu}
Celem projektu jest utworzenie systemu pozwalającego prowadzącemu zajęcia monitorować komputery w pracowni laboratoryjnej. Aplikacja do pracy wymagać będzie skonfigurowania serwera aplikacji, połączonego siecią lokalną z komputerami w pracowni. 

Serwer będzie udostępniał serwis internetowy, do którego prowadzący zajęcia będzie mógł zalogować się za pomocą przeglądarki internetowej. Aplikacja będzie udostępniała podgląd pulpitów i informacji o procesach uruchomionych na stanowiskach komputerowych. W domyślnym widoku będzie podgląd pewnej ilości ekranów, każdy w oddzielnej komórce siatki (najprawdopodobniej 5x4) -- podobnie jak w systemach monitoringu wizyjnego CCTV. Po wybraniu odpowiedniego stanowiska, program wyświetli podgląd z pulpitu danego komputera w wyższej jakości oraz zostaną wyświetlone pobierane przez aplikację kliencką dane. Będą to między innymi: uruchomione aplikacje, otwarte karty w przeglądarkach komputerowych, obecne użycie procesora i pamięci. W czasie, gdy prowadzący będzie zalogowany na swój panel, w zależności od konfiguracji będą mogły pojawiać się komunikaty, np. o wejściu studenta na stronę lub uruchomieniu programu z black listy lub takiej która nie znajduje się na white liście. 

Możliwe także, że system będzie umożliwiał zapisywanie logów, w których znajdą się takie informacje jak: połączenie i odłączenie stanowiska od systemu, próby zmiany konfiguracji oprogramowania klienta oraz komunikaty dla administratora.



% moja propozycja: white lista tylko dla kart przeglądarek; w procesach idzie się za***ć z definicją WL-ki. //N 
%jestem za //M
