\section {Wymagania funkcjonalne}
W poniższej sekcji zostaną przedstawione wymagania funkcjonalne. Najpierw natomiast muszą zostać przedstawieni aktorzy, których lista i opis znajdują się w poniższej tabeli.
\newline
\begin{tabular}{|c|c|} \hline
aktor & opis aktora \\ \hline
użytkownik   & jest to użytkownik, który obsługuje stanowisko komputerowe\\
             & w pracowni, nie musi on być nawet świadomy, że jest aktorem\\
             & w tym systemie \\ \hline

administrator   & jest to użytkownik, który ma możliwość zalogowania się \\
                & na stronie, która znajduje się na serwerze i umożliwia \\
                & podgląd stanowisk połączonych z serwerem. \\ \hline

\end{tabular}
\newline\newline
Po zapoznaniu się z krótkimi opisami dotyczących poszczególnych aktorów występujących w systemie
można zapoznać się z wymaganiami funkcjonalnymi. Te wymagania znajdują się w poniższej tabeli.
\newline\newline

\begin{tabular}{|c|c|c|c|} \hline
l.p. & funkcjonalność & aktor & obszar \\ \hline

1.  & robienie zrzutów ekranu w określonej          & użytkownik & aplikacja \\
    & częstotliwości i przesyłanie ich do serwera   &            & stanowiska \\ \hline
    
2.  & konfiguracja aplikacji stanowiska, wymaga     & administrator & aplikacja \\
    & wcześniejszego wejścia do programu oraz       &               & stanowiska \\
    & zalogowania się na konto administratora   && \\ \hline
    
3.  & podgląd wielu stanowisk jednocześnie, gdzie    & administrator& serwis \\
    & widok każdego stanowiska jest małym prostąkątem&              & internetowy \\
    & znajdującym się obok innych na ekranie         & & \\ \hline
    
4.  & podgląd w lepszej jakości stanowiska oraz       & administrator & serwis \\
    & listy uruchomionych aplikacji oraz kart i okien & & internetowy \\
    & w przeglądarkach internetowych & & \\ \hline
    
5.  & możliwość włączenia i konfiguracji jednej z    & administrator & serwis \\
    & dwóch list: whitelisty lub blacklisty stron    & & internetowy \\
    & internetowych na które może lub nie może      && \\
    & wchodzić użytkownik stanowiska komputerowego && \\ \hline
    
6. & wyświetlanie komunikatów o użytkownikach   & administrator & serwis \\
    & wchodzą na strony spoza whitelisty (jeśli & & internetowy \\
    & to ona jest wybrana) lub strony, która && \\
    &znajduje się na blackliście (jeśli wybrano && \\
    & drugą opcję) &&\\ \hline
    
7.  & oglądanie logów serwera oraz poszczególnych& administrator & serwis \\
    & stanowisk komputerowych && internetowy \\ \hline
    

\end{tabular}
